\documentclass[a4paper,12pt]{article}
\usepackage{cmap}
\usepackage[T2A]{fontenc}
\usepackage[utf8]{inputenc}
\usepackage[english,russian]{babel}

\title{Отчет об игре}
\author{Дзюбенко Мария Вячеславовна }
\date{Декабрь 2023}

\begin{document}

\maketitle

\section{Как же я сделала игру?}
Для начала я установила приложение Gogot,прочитав инструкцию которую вы нам отправили я начала писать код для игры,чтобы она работала.Не буду врать,я так же пользовалась видео-уроками на ютубе.Сначала код был написан для заставки(PLAY и QUIT).Далее для заставки в виде деревьев,а дальше я приступила к созданию персонажей и действий которые они должны выполнять.Для первого раза я считаю,что получилось не плохо.Это был интересный,но сложный опыт,так как в некоторых моментах у меня слетали коды и приходилось начинать заново.Я скажу вам больше,я переделывала игру 3 раза,так как не могла найти ошибку из-за которой то или иное действие пермонажей(и не только) просто не работало.Самым сложным для меня оказалось добавление эффекта Parallax,его я делала 4 часа...Да и переделывала я его раза 3,потому что деревья изначально не передвигались и это было проблемой.После того как я сделала заставку в виде деревьей,я сделала персонажей(в принципе это было главной составляющей игры).После код был написан для анимации игры.А дальше создание Моба и его анимация.Дальше были сделали не менее важные действия и в конце я уже сделала подсчет дней.Я бы могла и продолжить дальше усовершенствовать игру,но признаюсь честно...Нервов просто не хватило.Поэтому вышло как-то так.Спасибо за внимание!


\end{document}
